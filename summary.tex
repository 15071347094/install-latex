\chapter{总结和展望}

本文是个人最近一段时间的使用总结, 其中难免有不甚合理或晦涩难懂的部分. 
若用户在阅读本文档的过程中有任何意见和建议, 请发邮件或在 GitHub 中提 issue.

有用户指出,
WSL 中安装字体比较麻烦.
这里引用 \href{https://www.jianshu.com/p/e7f12b8c8602}{Ubuntu系统字体命令和字体的安装} 一文,
希望能够提供一些帮助.

首先获取需要安装的字体文件,
假设文件保存在 \verb|~/fonts/|.
然后在 \texttt{/usr/share/fonts/} 文件夹中创建新的文件夹,
例如 \texttt{myfonts}
\begin{lstlisting}[language=bash]
  cd /usr/share/fonts/
  sudo mkdir myfonts
\end{lstlisting}
接下来将获取的字体文件复制到 \texttt{myfonts} 中
\begin{lstlisting}[language=bash]
  sudo cp ~/fonts/* /usr/share/fonts/myfonts/ 
\end{lstlisting}
然后修改字体文件的权限
\begin{lstlisting}[language=bash]
  sudo chmod -R 755 myfonts
\end{lstlisting}
最后建立字体缓存
\begin{lstlisting}[language=bash]
  mkfontscale
  mkfontdir
  fc-cache -fv
\end{lstlisting}

在目前的版本中, 本文使用 WSL 方法进行编译.
目前感受是该方法较之 Windows 10 系统中的编译更快.
其他方面并未进行比较.
希望阅读本文的用户能够尽快上手.
