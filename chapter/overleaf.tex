% !TeX root = ../main.tex

\chapter{Overleaf}

在特定场合,
有些用户并不需要也没条件在本地安装发行版,
因此这里额外补充 \href{www.overleaf.com}{Overleaf} 的相关内容.

\section{注册 Overleaf}

Overleaf 是全球范围内首屈一指的在线 \LaTeX{} 编辑平台.
它为每位用户提供了 Ubuntu 系统下的 \TeX{} Live.
它优秀的协作功能、丰富的模板仓库已吸引全球科研工作者成为它的用户.
2019年9月23日,
它将后台的 \TeX{} Live 升级为 2018 版本,
并且为用户提供了思源宋体和思源黑体这两种开源中文字体.

遗憾的是,
国内网络环境会对 Overleaf 所使用的 reCaptcha 造成影响.
这也使得很多用户在直接注册 Overleaf 时就遇到了问题.

目前,
比较好的替代方案是借助 \href{https://orcid.org}{ORCID} 来进行注册.
目前使用国内网络访问 ORCID 还比较流畅.
用户, 尤其是科研工作者, 可以先注册一个 ORCID 账号.
未来投稿时,
可将 ORCID 账号与自己的期刊网站账号进行绑定.
同时,
用户也可逐步将自己所发表论文列在 ORCID 网站以便管理.

\section{使用 Overleaf}

用户通过 ORCID 注册 Overleaf 后便可进入自己的项目列表页面进行使用.

新建项目是用户首先使用的功能.
Overleaf 提供了多种渠道为用户新建项目:
可以通过 Overleaf 中的模板,
也可以通过用户本地的 \textsf{zip} 压缩文件包.

新建项目后,
用户便可进入编辑界面.
在编辑界面
用户需要先在左上角 \menu{Menu} 中选择合适的编译命令.
由于默认字体等原因,
强烈建议用户在处理中文文档时使用 \texttt{XeLaTeX}.
编写文档后,
用户可通过鼠标点击按钮进行编译,
也可使用快捷键 \keys{ctrl + enter}.

用户编写的文件会保存在网站.
编写完成后,
用户只需点击 \menu{Menu} 旁的箭头回到项目列表.
这时可以看到新增项目右侧有三个图标,
它们分别是 \menu{Copy}、\menu{Download} 和 \menu{Archive}.
用户可根据自己的需求点击合适的图标.

\section{升级项目}

Overleaf 将后台 \TeX{} Live 升级为 2018 版本后,
用户新建项目默认使用 \TeX{} Live 2018,
而老项目还是使用 \TeX{} Live 2017.
如果用户打算使用 \TeX{} Live 2018,
只需将项目复选框选择,
而后在右上角位置找到 \menu{More > Make a copy},
这样创建的副本便可以使用 \TeX{} Live 2018.
如果用户是从 Github 上面导入的项目,
建议再次导入.

\section{拓展学习}

有关 Overleaf 的更多使用技巧,
用户可自行阅读官网上丰富的素材,
这里不继续展开.
