% !TeX root = ../main.tex

\chapter{有关 WSL 的一点补充内容}

有用户指出,
WSL 中安装字体比较麻烦.
这里引用 \href{https://www.jianshu.com/p/e7f12b8c8602}{Ubuntu系统字体命令和字体的安装} 一文.

首先获取需要安装的字体文件,
假设文件保存在 \verb|~/fonts/|.
然后在 \texttt{/usr/share/fonts/} 文件夹中创建新的文件夹,
例如 \texttt{myfonts}
\begin{lstlisting}[language=bash]
  cd /usr/share/fonts/
  sudo mkdir myfonts
\end{lstlisting}
接下来将获取的字体文件复制到 \texttt{myfonts} 中
\begin{lstlisting}[language=bash]
  sudo cp ~/fonts/* /usr/share/fonts/myfonts/ 
\end{lstlisting}
然后修改字体文件的权限
\begin{lstlisting}[language=bash]
  sudo chmod -R 755 myfonts
\end{lstlisting}
最后建立字体缓存
\begin{lstlisting}[language=bash]
  mkfontscale
  mkfontdir
  fc-cache -fv
\end{lstlisting}

在目前的版本中, 本文在 WSL 中进行编译.
目前感受是该方法较之 Windows 10 系统中的编译更快.
其他方面并未进行比较.
由于 \TeX studio 在 WSL 中无法正常正反搜索,
因此我使用了 VS Code 配合
\href{https://marketplace.visualstudio.com/items?itemName=ms-vscode-remote.remote-wsl}{Remote WSL}
和
\href{https://marketplace.visualstudio.com/items?itemName=James-Yu.latex-workshop}{\LaTeX{} Workshop}
来编写文档.
这里将个人使用的相关配置附上,
其中注释的部分是调用外部 SumatraPDF 阅读器的配置,
用户可参考 \href{https://wenda.latexstudio.net/article-5055.html}{SumatraPDF 阅读器该怎么设置——以 TeXStudio 和 VS Code 为例}来了解更多内容.
\begin{lstlisting}
  "latex-workshop.latex.tools": [
    {
      "name": "latexmkpdf",
      "command": "latexmk",
      "args": [
        "-synctex=1",
        "-interaction=nonstopmode",
        "-halt-on-error",
        "-file-line-error",
        "-pdf",
        "%DOCFILE%"
      ]
    },
    {
        "name": "latexmkxe",
        "command": "latexmk",
        "args": [
          "-synctex=1",
          "-interaction=nonstopmode",
          "-halt-on-error",
          "-file-line-error",
          "-pdfxe",
          "%DOCFILE%"
        ]
      },
  ],
  "latex-workshop.latex.recipes": [
    {
      "name": "latexmkpdf",
      "tools": [
        "latexmkpdf"
      ]
    },
    {
        "name": "latexmkxe",
        "tools": [
          "latexmkxe"
        ]
      },
  ],
  "latex-workshop.latex.autoBuild.run": "never",
  "latex-workshop.view.pdf.viewer": "tab"
  // "latex-workshop.view.pdf.viewer": "external",
  // "latex-workshop.view.pdf.ref.viewer": "external",
  // "latex-workshop.view.pdf.external.viewer.command": "<SumatraPDFROOT>/SumatraPDF.exe",
  // "latex-workshop.view.pdf.external.viewer.args": [
  //   "-inverse-search",
  //   "\"<VSCodeROOT>/Code.exe\" \"<VSCodeROOT>/resources/app/out/cli.js\" -r -g \"%f:%l\"",
  //   "%PDF%"
  // ],
  // "latex-workshop.view.pdf.external.synctex.command":"<SumatraPDFROOT>/SumatraPDF.exe",
  // "latex-workshop.view.pdf.external.synctex.args":[
  //   "-forward-search",
  //   "%TEX%",
  //   "%LINE%",
  //   "%PDF%",
  // ],
\end{lstlisting}
注意更换 \texttt{<SumatraPDFROOT>} 和 \texttt{<VSCodeROOT>}.
希望阅读本文的用户能够尽快上手使用 \LaTeX.
