\chapter{Ubuntu 18.04 系统}

\section{安装 \TeX{} Live}

这里只阐述如何用镜像安装.
为使用户顺利使用 \TeX{} Live 2019, 建议用户首先卸载从源内安装的 \TeX{} Live 的相关包.

下载 \href{http://mirrors.ctan.org/systems/texlive/Images/texlive2019.iso}{iso 镜像文件} 的方法如前所述, 可选择国内源以加快下载速度.
下载完毕后, 打开 \textsf{Terminal} 窗口, 执行以下命令
\begin{lstlisting}[language = bash]
  cd ~/Downloads
  md5sum texlive2019.iso
\end{lstlisting}
若显示
\begin{lstlisting}
  f13ffe81840bb37de855bf7445e1d29a  texlive2019.iso
\end{lstlisting}
则镜像文件下载正确.

接下来, 使用如下代码加载光盘镜像至 \texttt{\~{}/Downloads/texlive} 文件夹
\begin{lstlisting}[language = bash]
  mkdir ~/Downloads/texlive
  sudo mount ./texlive2019.iso ~/Downloads/texlive
\end{lstlisting}
接下来执行
\begin{lstlisting}[language = bash]
  sudo ~/Downloads/texlive/install-tl
\end{lstlisting}
进行安装.
在屏幕上应该能见到以下内容
\begin{lstlisting}
  ======================> TeX Live installation procedure <=====================
  
  ======>   Letters/digits in <angle brackets> indicate   <=======
  ======>   menu items for actions or customizations      <=======
  
   Detected platform: GNU/Linux on x86_64
   
   <B> set binary platforms: 1 out of 5
  
   <S> set installation scheme: scheme-full
  
   <C> set installation collections:
       40 collections out of 41, disk space required: 5845 MB
  
   <D> set directories:
     TEXDIR (the main TeX directory):
       /usr/local/texlive/2019
     TEXMFLOCAL (directory for site-wide local files):
       /usr/local/texlive/texmf-local
     TEXMFSYSVAR (directory for variable and automatically generated data):
       /usr/local/texlive/2019/texmf-var
     TEXMFSYSCONFIG (directory for local config):
       /usr/local/texlive/2019/texmf-config
     TEXMFVAR (personal directory for variable and automatically generated data):
       ~/.texlive2019/texmf-var
     TEXMFCONFIG (personal directory for local config):
       ~/.texlive2019/texmf-config
     TEXMFHOME (directory for user-specific files):
       ~/texmf
  
   <O> options:
     [ ] use letter size instead of A4 by default
     [X] allow execution of restricted list of programs via \write18
     [X] create all format files
     [X] install macro/font doc tree
     [X] install macro/font source tree
     [ ] create symlinks to standard directories
     [X] after install, set CTAN as source for package updates
  
   <V> set up for portable installation
  
  Actions:
   <I> start installation to hard disk
   <P> save installation profile to 'texlive.profile' and exit
   <H> help
   <Q> quit
  
  Enter command: 
\end{lstlisting}
这里强烈建议用户直接点击 \keys{I} 使用默认配置安装.
如果用户对于 Ubuntu 系统比较了解, 可以根据提示, 更改安装设置.
安装完毕后, 将加载的光盘镜像弹出
\begin{lstlisting}[language = bash]
  sudo umount ~/Downloads/texlive
\end{lstlisting}

安装完成后, 用户需要设置环境变量.
继续在 \textsf{Terminal} 中执行
\begin{lstlisting}[language = bash]
  sudo gedit ~/.bashrc
\end{lstlisting}
在打开的文件末尾添加
\begin{lstlisting}
  # Add Tex Live to the PATH, MANPATH, INFOPATH
  export PATH=/usr/local/texlive/2019/bin/x86_64-linux:$PATH
  export MANPATH=/usr/local/texlive/2019/texmf-dist/doc/man:$MANPATH
  export INFOPATH=/usr/local/texlive/2019/texmf-dist/doc/info:$INFOPATH
\end{lstlisting}
并保存退出.
这时再打开 \textsf{Terminal} 执行
\begin{lstlisting}[language=bash]
  tex -v
\end{lstlisting}
将显示
\begin{lstlisting}
  TeX 3.14159265 (TeX Live 2019)
  kpathsea version 6.3.1
  Copyright 2019 D.E. Knuth.
  There is NO warranty.  Redistribution of this software is
  covered by the terms of both the TeX copyright and
  the Lesser GNU General Public License.
  For more information about these matters, see the file
  named COPYING and the TeX source.
  Primary author of TeX: D.E. Knuth.
\end{lstlisting}
即为安装成功.

\section{卸载 \TeX{} Live}

如果要卸载从源内安装的 \TeX{} Live, 个人比较推荐使用 synaptic package manager.
\textsf{terminal} 中执行
\begin{lstlisting}[language = bash]
  sudo apt install synaptic
\end{lstlisting}
即可安装.
安装后打开, 搜索 \textsf{texlive} 即可看到与之相关的包, 右键标记以删除即可.

如果是从光盘镜像安装, 只需要直接删除文件夹即可.
可先在\textsf{terminal} 中执行
\begin{lstlisting}[language = bash]
  kpsewhich -var-value TEXMFROOT
\end{lstlisting}
来查询安装路径,
进而通过 \texttt{sudo rm -rf} 进行删除.
默认安装的用户直接运行
\begin{lstlisting}[language = bash]
  sudo rm -rf /usr/local/texlive/2019
  rm -rf ~/.texlive
\end{lstlisting}
当然删除后还要清理掉环境变量.

\section{跨版本升级 \TeX{} Live}
在 \href{https://www.tug.org/texlive/upgrade.html}{tug.org} 网站上提供了相应的方法.
但网站也声明:
默认情况下,
请通过执行新安装来获取新版本的 \TeX{} Live\footnote{原文是: By default, please get the new TL by doing a new installation instead of proceeding here.}.

\section{升级宏包}

首先在 \textsf{terminal} 中执行
\begin{lstlisting}[language = bash]
  sudo visudo
\end{lstlisting}
将
\begin{lstlisting}
  /usr/local/texlive/2019/bin/x86_64-linux:
\end{lstlisting}
添加在 \texttt{secure\_path} 中.
然后依次 \keys{\ctrl + X}, \keys{Y}, \keys{\enter} 保存退出.

接下来在 \textsf{terminal} 中执行
\begin{lstlisting}[language = bash]
  sudo tlmgr option repository ctan
\end{lstlisting}
更改源, \texttt{ctan} 也可以改成其他地址, 详情见 Windows 10 系统升级宏包部分.
接下来, 用户执行命令
\begin{lstlisting}[language = bash]
  sudo tlmgr update --list
\end{lstlisting}
可查看目前源上可升级的宏包都有哪些. 
高级用户可以根据自己的需求选择升级特定宏包, 而初级用户建议直接升级全部宏包. 
用户只需执行
\begin{lstlisting}[language = bash]
  sudo tlmgr update --self --all
\end{lstlisting}
同时升级 \texttt{tlmgr} 本身和全部宏包. 

\section{安装宏包}

Ubuntu 18.04 下安装宏包的要求与 Windows 10 下没有多少区别, 只需注意权限, 例如
\begin{lstlisting}[language = bash]
  sudo tlmgr install mcmthesis
\end{lstlisting}
即安装了 mcmthesis.

\section{编译文件}

首先, 用户需要在工作路径建立一个 \texttt{tex} 文件.
在 \textsf{Terminal} 中执行
\begin{lstlisting}[language = bash]
  mkdir ~/Documents/work_latex
  cd ~/Documents/work_latex/
  gedit main.tex
\end{lstlisting}
在打开的文件输入一个最小实例
\begin{lstlisting}[language = {[LaTeX]TeX}]
  \documentclass{article}
  \begin{document}
    Hello \LaTeX{} World!
  \end{document}
\end{lstlisting}
保存并退出. 
接下来执行
\begin{lstlisting}
  pdflatex main
\end{lstlisting}
等待系统完成编译过程. 
待编译完成后, 我们即可看到在 \texttt{\~{}/Documents/work\_latex} 中出现了 \texttt{main.pdf} 文件和其他同名的辅助文件 \texttt{main.aux} 与 \texttt{main.log}. 
执行
\begin{lstlisting}[language=bash]
  evince main.pdf
\end{lstlisting}
即可打开 \texttt{pdf} 文件.

编译命令可添加参数, 这里与 Windows 10 中的情形一致, 不再赘述.

\subsection{无法使用 \texttt{xelatex} 命令}

有些用户反映安装完毕后无法使用 \texttt{xelatex} 命令.
这里或许是因为 libfontconfig 缺失, 用户可在命令行中执行 
\begin{lstlisting}[language=bash]
  sudo apt-get install libfontconfig1
\end{lstlisting}
进行安装.

\section{使用编辑器}

简化起见, 这里只介绍如何使用 \TeX studio.

根据官网推荐, 我们安装源内的 \TeX studio.
由于网络问题, 直接安装速度比较慢, 因此我们首先更换 Ubuntu 18.04 的源至清华大学.
在 \textsf{Terminal} 中执行
\begin{lstlisting}[language = bash]
  sudo cp /etc/apt/sources.list /etc/apt/sources.list.bak
\end{lstlisting}
备份 \texttt{sources.list} 文件.
接下来执行
\begin{lstlisting}[language = bash]
  sudo gedit /etc/apt/sources.list
\end{lstlisting}
将文件替换为以下内容\footnote{本段文字可至 \href{https://mirrors.tuna.tsinghua.edu.cn/help/ubuntu/}{清华大学镜像网站} 获取}
\begin{lstlisting}
  # 默认注释了源码镜像以提高 apt update 速度, 如有需要可自行取消注释
  deb https://mirrors.tuna.tsinghua.edu.cn/ubuntu/ bionic main restricted universe multiverse
  # deb-src https://mirrors.tuna.tsinghua.edu.cn/ubuntu/ bionic main restricted universe multiverse
  deb https://mirrors.tuna.tsinghua.edu.cn/ubuntu/ bionic-updates main restricted universe multiverse
  # deb-src https://mirrors.tuna.tsinghua.edu.cn/ubuntu/ bionic-updates main restricted universe multiverse
  deb https://mirrors.tuna.tsinghua.edu.cn/ubuntu/ bionic-backports main restricted universe multiverse
  # deb-src https://mirrors.tuna.tsinghua.edu.cn/ubuntu/ bionic-backports main restricted universe multiverse
  deb https://mirrors.tuna.tsinghua.edu.cn/ubuntu/ bionic-security main restricted universe multiverse
  # deb-src https://mirrors.tuna.tsinghua.edu.cn/ubuntu/ bionic-security main restricted universe multiverse
  
  # 预发布软件源, 不建议启用
  # deb https://mirrors.tuna.tsinghua.edu.cn/ubuntu/ bionic-proposed main restricted universe multiverse
  # deb-src https://mirrors.tuna.tsinghua.edu.cn/ubuntu/ bionic-proposed main restricted universe multiverse
\end{lstlisting}
然后执行
\begin{lstlisting}[language = bash]
  sudo apt-get update && sudo apt-get upgrade
\end{lstlisting}
至此换源完毕.

在 \textsf{Terminal} 中执行
\begin{lstlisting}[language = bash]
  sudo apt install texstudio
\end{lstlisting}
即可安装 \TeX studio.
注意安装过程中会产生一些依赖, 它们包括了源内的 \TeX{} live 包, 如 \texttt{tex-common}, \texttt{texlive-base}, \texttt{texlive-binaries}, \texttt{texlive-latex-base} 和 \texttt{texlive-latex-recommended}.
用户需要卸载它们.

用户可以在 \textsf{Terminal} 中执行
\begin{lstlisting}[language = bash]
  texstudio main.tex
\end{lstlisting}
使用 \TeX studio 编辑文档.
也可直接双击 \texttt{main.tex} 文件.

注意, 在双击打开时, 用户需要在 \menu{Options > Configure TeXstudio} 中 点击 \menu{Show Advanced Options}, 接下来在 \menu{Build > Build Options > Commands (\${}PATH)} 中添加 \texttt{/usr/local/texlive/2019/bin/x86\_64-linux}.
