% !TeX document-id = {9518a27b-c930-4bf6-a805-aa54f9dee147}
% !TeX TXS-program:compile=txs:///latexmk/{}[-pdfxe -synctex=1 -interaction=nonstopmode -silent %]

\documentclass{ctexart}
\usepackage[margin=1in]{geometry}
\usepackage{listings}
\usepackage{xcolor}
\usepackage{hyperref}
\usepackage[os=win,hyperrefcolorlinks]{menukeys}

\ctexset{
  section/format = \Large\bfseries\raggedright
}

\lstset{
  backgroundcolor = \color{lightgray!30},
  keywordstyle = \color{blue},
  stringstyle = \color{purple!40},
  basicstyle = {\small\ttfamily},
  breaklines = true,
  tabsize = 4,
  gobble = 2,
  numbers = left,
  numberstyle = \tiny,
  emph={ 
    mkdir,tlmgr,pdflatex,latexmk,documentclass,begin,LaTeX
  },
  emphstyle={\color{blue}\small\ttfamily}%
}

\renewmenumacro{\menu}[>]{angularmenus}
\renewmenumacro{\keys}[+]{shadowedroundedkeys}

\title{\bfseries 一份简短的安装 \LaTeX{} 的介绍}
\author{啸行\thanks{\url{ranwang.osbert@outlook.com}}}

\begin{document}
  
\maketitle

\begin{abstract}
在 QQ 群 91940767 时,经常有群友咨询如何安装 \LaTeX{}。
本文将介绍 Windows 10 系统中安装 \TeX{}~Live、升级宏包、编译简易文档、使用编辑器的相关操作,并多以介绍命令行操作为主。
有关 Mik\TeX{} 的安装,可以参考 \href{https://camuseblog.top/2019-03-02-/MiKTeX/}{MiK\TeX{} 的基本使用}。
建议用户阅读 \href{http://www.latexstudio.net/archives/11469.html}{\LaTeXe{} 安装 \& 新手指点 FAQ} 和 \href{http://mirrors.ctan.org/info/lshort/chinese/lshort-zh-cn.pdf}{lshort-zh-ch} 以更全面地了解基础内容。
本文还将简要介绍两款常见编辑器,其他编辑器如 \href{https://github.com/EthanDeng/vscode-latex}{VS code} 和 \href{https://github.com/EthanDeng/sublime-text-latex}{Sublime Text},用户可自行了解它们的使用方法。
本文所涉及到的代码需结合上下文说明,不能简单地复制粘贴。
红色文字都是可点的超链接,可直接跳转。
\menu{菜单} 表示软件菜单。
\keys{k} 表示键盘按键。

\end{abstract}

\section{安装 \TeX{} Live}
首先用户下载 \TeX{} Live 的 \href{http://mirrors.ctan.org/systems/texlive/Images/texlive.iso}{iso 镜像文件}。
下载完毕后,可以将镜像文件加载至虚拟光驱\footnote{Windows 10 默认双击镜像文件便可加载,无需使用其他软件;在安装了解压缩软件后,镜像文件依然可以通过文件夹上方 \menu{主页 > 打开 > Windows 资源管理器} 加载;这里不鼓励解压缩的操作;第三方虚拟光驱软件可考虑 WinCDEmu 或 UltraISO}。
假设加载后的路径为 \texttt{X:\textbackslash texlive},其中 \texttt{X} 表示盘符,如 \texttt{C}、\texttt{D}、\texttt{E} 等。

接下来,用户打开 \textsf{cmd} 窗口\footnote{在 Windows 搜索栏里面搜 cmd,系统显示“命令提示符”即为 \textsf{cmd} 窗口},执行以下代码
\begin{lstlisting}[language = bash]
  echo %path%
\end{lstlisting}
查看环境变量。
若 \texttt{C:\textbackslash Windows\textbackslash system32} 不在结果中\footnote{这里默认系统盘为\textsf{C}盘},则关闭 \textsf{cmd} 窗口,将 \texttt{C:\textbackslash Windows\textbackslash system32} 添加到环境变量中\footnote{在 \menu{桌面 > 此电脑 > 属性 > 高级系统设置 > 环境变量 > 系统变量 > Path} 中添加}。
接下来,在关闭了国内第三方安全软件的前提下,再打开 \textsf{cmd} 窗口并执行
\begin{lstlisting}[language = bash]
  cd /d X:\texlive
\end{lstlisting}
切换路径。
接下来执行\footnote{强烈建议用户在操作系统中显示文件后缀名,文件夹上方 \menu{查看 > 显示/隐藏} 中选择 \menu{文件扩展名}}
\begin{lstlisting}[language = bash]
  install-tl-windows.bat --no-gui
\end{lstlisting}
即可进入安装流程。

此安装方法所采取的是纯命令行操作,因此用户将看到如下的结果\footnote{有部分用户反应,使用命令行安装时,系统会提示用户无权限(Permission),对于这种情况,可以右键 \menu{命令提示符},选择 \menu{以管理员身份运行}}
\begin{lstlisting}
  =====================> TeX Live installation procedure <=====================
  
  ======>   Letters/digits in <angle brackets> indicate   <=======
  ======>   menu items for actions or customizations      <=======
  
  Detected platform: Windows
  
  <B> set binary platforms: 1 out of 8
  
  <S> set installation scheme: scheme-full
  
  <C> set installation collections:
  41 collections out of 41, disk space required: 5541 MB
  
  <D> set directories:
  TEXDIR (the main TeX directory):
  C:/texlive/2018
  TEXMFLOCAL (directory for site-wide local files):
  C:/texlive/texmf-local
  TEXMFSYSVAR (directory for variable and automatically generated data):
  C:/texlive/2018/texmf-var
  TEXMFSYSCONFIG (directory for local config):
  C:/texlive/2018/texmf-config
  TEXMFVAR (personal directory for variable and automatically generated data):
  ~/.texlive2018/texmf-var
  TEXMFCONFIG (personal directory for local config):
  ~/.texlive2018/texmf-config
  TEXMFHOME (directory for user-specific files):
  ~/texmf
  
  <O> options:
  [ ] use letter size instead of A4 by default
  [X] allow execution of restricted list of programs via \write18
  [X] create all format files
  [X] install macro/font doc tree
  [X] install macro/font source tree
  [X] adjust search path
  [1] add menu items, shortcuts, etc.                                                                                                                                                                                                          [1] update file associations                                                                                                                                                                                                                 [X] install TeXworks front end
  [X] after install, use tlnet on CTAN for package updates
  
  <V> set up for portable installation
  
  Actions:
  <I> start installation to hard disk
  <P> save installation profile to 'texlive.profile' and exit
  <H> help
  <Q> quit
  
  Enter command:
\end{lstlisting}
这时,用户可先执行 \texttt{D} 来更改安装路径,也可直接执行 \texttt{I} 在默认路径中直接安装 \TeX{} Live。
执行 \texttt{D} 后,用户可看到
\begin{lstlisting}
  ==============================================================================
  Directories customization:
  
  <1> TEXDIR:       C:/texlive/2018
  support tree: C:/texlive/2018/texmf-dist
  
  <2> TEXMFLOCAL:     C:/texlive/texmf-local
  <3> TEXMFSYSVAR:    C:/texlive/2018/texmf-var
  <4> TEXMFSYSCONFIG: C:/texlive/2018/texmf-config
  
  <5> TEXMFVAR:       ~/.texlive2018/texmf-var
  <6> TEXMFCONFIG:    ~/.texlive2018/texmf-config
  <7> TEXMFHOME:      ~/texmf
  
  Note: ~ will expand to %USERPROFILE%
  
  Actions:
  <R> return to main menu
  <Q> quit
  
  Enter command:
\end{lstlisting}
这时执行数字 \texttt{1},将看到
\begin{lstlisting}
  New value for TEXDIR [C:/texlive/2018]:
\end{lstlisting}
即可更改路径,如 \texttt{D:/texlive/2018}。
接下来,执行 \texttt{R} 即可回到初始的安装界面。
其他操作用户可通过阅读英文自行执行,这里不再赘述。
至此,用户只需耐心等待安装完成。
安装完成后,关闭再打开 \textsf{cmd} 窗口,执行
\begin{lstlisting}
  tex -v
\end{lstlisting}
以验证安装是否成功\footnote{正常安装 \TeX{} Live 后,环境变量会变化,关闭原 \textsf{cmd} 窗口再重新打开才能获取新的环境变量}。
结果显示为
\begin{lstlisting}
  TeX 3.14159265 (TeX Live 2018/W32TeX)
  kpathsea version 6.3.0
  Copyright 2018 D.E. Knuth.
  There is NO warranty.  Redistribution of this software is
  covered by the terms of both the TeX copyright and
  the Lesser GNU General Public License.
  For more information about these matters, see the file
  named COPYING and the TeX source.
  Primary author of TeX: D.E. Knuth.
\end{lstlisting}
即可认为安装顺利完成 \footnote{有用户反应安装后环境变量未发生变化,则可手动添加 \texttt{<TEXDIR>\textbackslash bin\textbackslash win32} 至环境变量,\texttt{<TEXDIR>} 需与你方才的安装地址一致}。

\section{卸载 \TeX{} Live}
有些时候,如跨版本升级 \TeX{} Live, 用户需要卸载 \TeX{} Live。
相较于安装,卸载稍容易一些。
用户找卸载批命令文件,如 \texttt{C:\textbackslash texlive\textbackslash 2018\textbackslash tlpkg\textbackslash installer\textbackslash uninst.bat}。
双击或在 \textsf{cmd} 中运行均可。

\section{升级宏包}
安装完成后,用户可以升级宏包以获得更好的使用体验。
下面将介绍使用命令行升级宏包的方法。
打开 \textsf{cmd} 窗口,首先执行下面命令指定升级使用的镜像源。
\texttt{ctan} 表示系统在升级时将自动寻求最近的源进行下载。
\begin{lstlisting}
  tlmgr option repository ctan
\end{lstlisting}
用户同样可以指定其他的镜像源,比如清华大学的镜像源:
\begin{lstlisting}
  tlmgr option repository http://mirrors.tuna.tsinghua.edu.cn/CTAN/systems/texlive/tlnet
\end{lstlisting}
或者中国科技大学的镜像源:
\begin{lstlisting}
  tlmgr option repository https://mirrors.ustc.edu.cn/CTAN/systems/texlive/tlnet/
\end{lstlisting}
接下来,用户执行命令
\begin{lstlisting}
  tlmgr update --list
\end{lstlisting}
可查看目前源上可升级的宏包都有哪些。
高级用户可以根据自己的需求选择升级特定宏包,而初级用户建议直接升级全部宏包。
用户只需执行
\begin{lstlisting}
  tlmgr update --self --all
\end{lstlisting}
同时升级 \texttt{tlmgr} 本身和全部宏包。

部分用户会碰到升级不成功的情况,这些情况我们将一一给出处理办法。

\subsection{\texttt{tlmgr} 本身无法成功升级}

遇到这种情况时,用户需自行下载 \href{http://mirror.ctan.org/systems/texlive/tlnet/update-tlmgr-latest.exe}{update-tlmgr-latest.exe},然后再执行升级命令即可。

\subsection{升级到一半停止}

这种情况下,用户需要执行另外一个命令继续升级
\begin{lstlisting}
  tlmgr update −−reinstall−forcibly−removed --all
\end{lstlisting}

\section{安装宏包}
在默认状态下,用户将完整安装 \TeX{} Live,因此用户极少碰到需要手动安装宏包的情形。
同时,在 \href{http://mirrors.ctan.org/info/lshort/chinese/lshort-zh-cn.pdf}{lshort-zh-ch} 中也明确提到,\textbf{如非万不得已,尽量不要手动安装宏包}。
因此在这里我只介绍从源处安装宏包的命令。
假设用户想安装 \texttt{mcmthesis} 宏包,只需在 \textsf{cmd} 窗口执行
\begin{lstlisting}
  tlmgr install mcmthesis
\end{lstlisting}
需要注意的是,用户一定要清楚所要安装的宏包名称,并且在安装宏包前先确保镜像源设置正确。

\section{编译文档}
升级宏包完成后,用户可以编译文档。
在这里,我依旧执拗地使用命令行来完成这一过程。

首先,用户需要在指定位置\footnote{以下称工作路径}建立一个 \texttt{tex} 文件:
\begin{lstlisting}[language = bash]
  mkdir E:\work_latex
  cd /d E:\work_latex
  notepad main.tex
\end{lstlisting}
第1行表示创建一个工作路径 \texttt{E:\textbackslash work\_latex},第2行表示进入工作路径,第3行表示用记事本打开 \texttt{main.tex} 文件,若文件不存在,系统将询问用户是否创建该文件。
在打开的记事本中输入一个最小实例
\begin{lstlisting}[language = TeX]
  \documentclass{article}
  \begin{document}
    Hello \LaTeX{} World!
  \end{document}
\end{lstlisting}
保存并退出。
接下来执行
\begin{lstlisting}
  pdflatex main
\end{lstlisting}
等待系统完成编译过程。
待编译完成后,我们即可看到在 \texttt{E:\textbackslash work\_latex} 中出现了 \texttt{main.pdf} 文件和其他同名的辅助文件 \texttt{main.aux} 与 \texttt{main.log}。

总结一下,在使用 \LaTeX{} 时,用户首先要在工作路径\footnote{建议不同的工程放入不同的工作路径}中编写好一个 \texttt{tex} 文件。
当 \texttt{tex} 文件编写完毕,我们即可使用编译命令\footnote{目前常用的编译命令为 \texttt{pdflatex} 和 \texttt{xelatex}},将 \texttt{tex} 文件编译成 \texttt{pdf} 文件。

编译命令其实有很多可选参数,如用户能够从 \texttt{pdf} 文件跳回 \texttt{tex} 文件,便是因为执行编译命令时添加了参数
\begin{lstlisting}
  pdflatex -synctex=1 main
\end{lstlisting}
此外还有很多其他参数,用户若感兴趣,可自行阅读手册。

在很多时候,我个人更倾向于使用 \texttt{latexmk} 来编译 \texttt{tex} 文档,如执行
\begin{lstlisting}
  latexmk -pdf -synctex=1 -interaction=nonstopmode main
\end{lstlisting}
意味着使用 
\begin{lstlisting}
  pdflatex -synctex=1 -interaction=nonstopmode main
\end{lstlisting}
来编译文档 \texttt{main.tex},并在有需要时完成其他步骤\footnote{如使用 \texttt{bibtex} 或 \texttt{biblatex} 处理参考文献时需要多次编译,详情见相关文档}。

\section{使用编辑器}
在实际操作中,用户会发觉使用记事本编写 \texttt{tex} 文件十分不便,因此很多用户都将其他编辑器作为自己的首选。
这里向大家介绍随 \TeX{} Live 一同发行的轻量级编辑器 TeXworks 和功能更加丰富的 \href{https://github.com/texstudio-org/texstudio/releases}{TeXstudio}。
其他编辑器,因作者能力有限,无法一一列举,还请自己在网上寻求更多帮助。

\subsection{TeXworks}
TeXworks 是一款轻量级的 \LaTeX 编辑器,非常适合入门级用户使用\footnote{很多用户在编写 \texttt{tex} 文件时喜欢借助其他工具,这不是个好习惯,这里鼓励用户多手动敲代码}。

和大多数国产软件不同,TeXworks 不会自动在桌面生成快捷方式,新人往往不知道如何打开它。
实际上 TeXworks 在 \TeX{} Live 安装路径的 \texttt{\textbackslash bin\textbackslash win32} 中,可在 \textsf{cmd} 中执行 \texttt{texworks} 打开,也可直接在 Windows 搜索栏里搜 texworks 打开。

刚开始使用 TeXworks 的用户不必过多配置,在默认状态下打开软件、编写代码、保存文件后,用户可使用 TeXworks 的“排版”按钮进行文档编译而不必回到命令行\footnote{实际上编辑器的按钮也是调用命令行来编译文档,TeXworks 默认排版快捷键是 \texttt{ctrl + T}}。
按钮旁边是编译命令菜单,用户可根据需要自行选择。

下面介绍几个常用的功能。

\subsubsection{显示行号}
打开 \menu{编辑 > 首选项 > 编辑器},选择 \menu{行号} 即可。

\subsubsection{自动补全}
自动补全主要是 \keys{tab} 键的功能,使用前需确定 \menu{首选项 > 编辑器} 中是否勾选了 \menu{启动自动补全}。
有关自动补全的更多内容用户可参考 \href{https://github.com/EthanDeng/texworks-autocomplete}{TeXworks 的自动补全功能}。

\subsubsection{使用模板}
TeXworks支持从模板新建文档,在 \menu{文件} 菜单中可见。

\subsubsection{拼写检查}
默认情况,Texworks没有搭载拼写检查字典,需要用户自己配置。
首先,访问 \href{http://wiki.openoffice.org/wiki/Dictionaries}{openoffice} 下载词典。
接下来将词典安装到 \texttt{<resources>\textbackslash dictionaries} \footnote{\texttt{<resources>} 的具体位置可以在 \menu{帮助 > TeXworks 配置与资源} 中找到}。
安装完成后,在 \menu{编辑 > 拼写} 中选择即可。

\subsubsection{魔法注释}
魔法注释可以直接规定主文档、文件编码和编译命令等。
例如用户写学位论文时,往往有一个主文档和若干子文档。
在子文档的开头写上魔法注释
\begin{lstlisting}
  % !TeX root = ../mainfile.tex
\end{lstlisting}
即可告诉系统上一级目录中的 \texttt{main.tex} 是主文档\footnote{主文档和子文档的更多内容请参考 \texttt{\textbackslash include} 和 \texttt{\textbackslash input} 的用法}。
\begin{lstlisting}
  % !TeX encoding = UTF8
\end{lstlisting}
表示该文档使用 ``UTF-8'' 编码\footnote{目前建议用户多用 ``UTF-8'' 编码,尤其是中文文档}。
\begin{lstlisting}
  % !TeX program = xelatex
\end{lstlisting}
表示该文档使用 \texttt{xelatex} 编译命令进行编译。
\begin{lstlisting}
  % !TeX spellcheck = <language_code>
\end{lstlisting}
表示该文档使用 \texttt{<language\_code>} 进行拼写检查,它的具体名称请用户到 \menu{编辑 > 拼写} 中查看。

\subsubsection{转编码}
实际上转编码是很多编辑器都具备的功能。
在TeXworks右下角有三个按钮,左边按钮控制换行符类型,中间按钮控制文档编码,右边按钮控制行跳转。
目前大部分中文用户主要面临的是``UTF-8''和``GBK''之间的转换。
如果文档是``GBK''编码,使用TeXworks打开文件后,文档会出现乱码。
这时,点中间按钮,选择编码类型``GBK'',再点击按钮,选择 \menu{使用所选编码重载文档},若文档中乱码消失,则再次点击按钮,选择``UTF-8'',最后保存文档,完成转码工作。

除以上内容外,TeXworks 还支持正则表达式、自定义快捷键等。
这些内容都写在 TeXworks 自带的手册中。
手册不长,用户可以通读一遍以了解更多内容。

\subsection{TeXstudio}
相较于 TeXworks,TeXstudio 功能更丰富,用法更多。

\subsubsection{更改默认编译命令}
在 \menu{Options > configure TeXstudio > build > default compiler} 中选择默认编译命令。

另外如果有用户喜欢用 \texttt{latexmk} 和 \texttt{biblatex},可以考虑在 \menu{Options > configure TeXstudio > build > Build Options} 中取消 \menu{Check and update bibliography before compiling}。

\subsubsection{显示行号}
在 \menu{Options > configure TeXstudio} 中勾选 \menu{Show Advanced Options},然后在 \menu{Adv. Editor > show line numbers} 选择。

\subsubsection{调整缩进}
在 \menu{Options > Editor > Indentation Mode} 处可以选择是否保持缩进,勾选 \menu{Replace Indentation Tab by Spaces} 可以让缩进完全变为空格。

在 \menu{Options > Adv. Editor > Appearance} 中可以更改 \menu{Tab Widths}。

\subsubsection{自动补全}
TeXstudio 自动补全功能会先通过用户引用的宏包来判断代码是否需要自动补全,而它的自动补全内容全部被写入了 \href{https://github.com/texstudio-org/texstudio/tree/master/completion}{cwl 文件} 中。
用户可以根据自己的需要更改 cwl 文件,然后将其放入 \texttt{<settings directory>\textbackslash completion\textbackslash user} 文件夹 \footnote{点击 \menu{Help > Check LaTeX Installation},在生成的 \texttt{System Report.txt} 中找到 \texttt{texstudio.ini} 文件,其所在文件夹即为 \texttt{<settings directory>}}。
自动补全后有可能生成文本框,使用 \keys{\ctrl + \arrowkey{>}} 可以跳转至下一个文本框;\keys{\ctrl + \arrowkey{<}} 则可以跳转至下一个。

\subsubsection{拼写检查}
TeXstudio 默认使用德语进行拼写检查,在右下角将 \texttt{de\_DE} 改为 \texttt{en\_US} 即可进行英文拼写检查,

\subsubsection{魔法注释}
除了像 TeXworks 中使用魔法注释外,TeXstudio 还有很多其他的魔法注释,例如
\begin{lstlisting}
  % !TeX TXS-program:compile = txs:///latexmk/{}[-xelatex -synctex=1 -interaction=nonstopmode %.tex]
\end{lstlisting}
表示使用 \texttt{latexmk} 编译命令,\texttt{\{\}} 表示无视编辑器赋予该命令的一切参数,而 \texttt{[]} 表示添加其中的参数进行编译,本例中添加 \texttt{-xelatex -synctex=1 -interaction=nonstopmode} 作为参数。

\subsubsection{渲染方式}
TeXstudio 默认渲染方式对中文括号的支持不够好。
在 \menu{Options > Configure TeXstudio > Adv. Editor > Hacks/Workarounds} 中取消 \menu{Try to automatically choose best display options},选择 \menu{Disable cache of character width} 和 \menu{Disable fixed pitch mdoe}。

\subsubsection{转编码}
TeXstudio右下角有一处显示文档编码 (Encoding)。
依旧以``UTF-8''和``GBK''之间的转换为例。
如果文档是``GBK''编码,使用TeXstudio打开文件后,点击 \menu{Encoding} 处,选择 \menu{More Encodings}。
打开窗口后,选择``GBK / ...'',点击 \menu{Reload With},若这时文档没有乱码,再点击 \menu{Encoding > More Encodings},选择``UTF-8'',点击 \menu{Change To},保存文件,完成转码。

\subsubsection{自定义命令并生成按钮}
TeXstudio 允许用户自定义命令,并将命令做成按钮放置于面板上。
例如,在 \menu{Options > Configure TeXstudio > Build > User commands} 中,填写名称 \texttt{User1:Build-xe-view} 和功能 \texttt{latexmk -pdfxe -silent -synctex=1 -interaction=nonstopmode \% | txs:///view-pdf-internal},点击 \menu{OK}。
接下来我们在 \menu{Tools > User} 中即可看到自定义的命令和相应快捷键。
再打开 \menu{Options > Configure TeXstudio > Toolbars},在两个下拉菜单中分别选择 \menu{Tools} 和 \menu{All menus},在右边找到 \menu{Tools > User > Build-xe-view} 并将其添加至左端,点击 \menu{OK}。
这时,在面板中将添加新的按钮。

其他有关 \texttt{texstudio} 的用法,用户也可以参阅手册进行学习。

\section{总结}
本文是个人最近一段时间的使用总结,其中难免有不甚合理或晦涩难懂的部分。
若用户在阅读本文档的过程中有任何意见和建议,请发邮件或在 GitHub 中提 issue\footnote{\url{https://github.com/OsbertWang/install_latex}}。

\end{document}
