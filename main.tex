\documentclass{ctexart}
\usepackage[margin=1in]{geometry}
\usepackage{listings}
\usepackage{xcolor}
\usepackage[colorlinks]{hyperref}

\lstset{
  backgroundcolor = \color{lightgray!30},
  keywordstyle = \color{blue},
  commentstyle = \color{green!40},
  stringstyle = \color{purple!40},
  basicstyle = {\small\ttfamily},
  breaklines = true,
  tabsize = 4,
  gobble = 2
}

\title{一份简短的安装 \LaTeX 的介绍}
\author{啸行\thanks{\url{ranwang.osbert@outlook.com}}}

\begin{document}
  
\maketitle

\begin{abstract}
  在QQ群91940767时,经常有群友咨询如何安装 \LaTeX。本文将介绍Windows系统中安装 \TeX Live、升级宏包、编译简易文档、配置编辑器的相关操作,并多以介绍命令行操作为主。建议用户阅读 \href{http://www.latexstudio.net/archives/11469.html}{LaTeX2e安装 \& 新手指点 FAQ} 和 \href{http://mirrors.ctan.org/info/lshort/chinese/lshort-zh-cn.pdf}{lshort-zh-ch} 以更全面地了解基础内容。本文所涉及到的代码还请用户多多动手,不要简单地复制粘贴。
\end{abstract}

\section{安装 \TeX Live}
首先用户下载 \TeX Live 的 \href{http://mirrors.ctan.org/systems/texlive/Images/texlive.iso}{光盘镜像}。下载完毕后,可以将光盘镜像加载至虚拟光驱\footnote{Windows 10 默认双击镜像文件便可加载,无需使用其他软件},亦可直接解压缩。假设加载或解压缩后的路径为 \verb|X:\texlive|。

接下来,用户打开 \verb|cmd| 窗口\footnote{微软键加\texttt{R}键,即可打开“运行”,在其中输入\texttt{cmd}后确定即可},执行\footnote{输入下列代码并按\texttt{enter}键即为执行}
\begin{lstlisting}[language = bash]
  echo %path%
\end{lstlisting}
查看环境变量\footnote{注意空格}。若 \verb|C:\Windows\system32| 不在结果中\footnote{这里默认系统盘为\texttt{C}盘},则需要在 \verb|cmd| 中继续执行
\begin{lstlisting}[language = bash]
  path=C:\Windows\system32;%path%
\end{lstlisting}
将 \verb|C:\Windows\system32| 添加到环境变量中。

接下来,在关闭了国内第三方安全软件的前提下,在\texttt{cmd}窗口中执行
\begin{lstlisting}[language = bash]
  cd /d X:\texlive
\end{lstlisting}
切换路径。接下来执行
\begin{lstlisting}[language = bash]
  install-tl-windows.bat -no-gui
\end{lstlisting}
即可进入安装流程。

\end{document}